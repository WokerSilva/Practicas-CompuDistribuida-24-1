\documentclass[a4paper,12pt]{article} 
\usepackage[utf8]{inputenc} % Acentos válidos sin problemas
\usepackage[spanish]{babel} % Idioma
\input{packet}


\begin{document}%----------------------INICIO DOCUMENTO------------|
%------------------------------------------------------------------|

\newpage

\input{portada}

\newpage

\begin{center}
    {\Large Algotitmo Dijkstra Distribuido}
\end{center}

\section*{Forma de compilar}

\section*{Funcionamiento}

\begin{itemize}
   \item[1.] Definición de Estructuras de Datos:
   \begin{itemize}
      \item[//] Se definen tres estructuras de datos:
      \begin{itemize}
         \item struct vertice: Representa un vértice en el grafo con un identificador único (id) y un valor de retraso (retraso).
         \item struct arista: Representa una arista (conexión) entre dos vértices con información sobre el vértice de origen (origen), el vértice de destino (destino) y el peso de la arista (peso).
         \item struct grafica: Representa el grafo en sí y contiene el número de vértices (nvertices), un array de vértices (vertices) y un array de aristas (aristas).
      \end{itemize}
   \end{itemize}

   \item[2.] Inicialización del Grafo ($inicio_grafica$):
   \begin{itemize}
      \item[//] Esta función inicializa el grafo con un número especificado de vértices (20 en este caso).
      \item[//] Asigna memoria dinámicamente para los arrays de vértices y aristas.
      \item[//] Inicializa los vértices con valores predeterminados, donde cada vértice tiene un identificador único y un retraso inicial de 0.
      \item[//] Crea aristas entre vértices consecutivos y asigna retrasos aleatorios a las aristas. El retraso aleatorio se genera entre 1 y 1000.
   \end{itemize}

   \item[3.] Impresión de la Información del Grafo ($imprimir_grafica$):
   \begin{itemize}
      \item[//] Esta función imprime la información del grafo en forma de tabla, mostrando el ID de cada vértice y su retraso.
   \end{itemize}

   \item[4.] Algoritmo de Dijkstra (dijkstra):
   \begin{itemize}
      \item[//] Esta función implementa el algoritmo de Dijkstra para encontrar las distancias más cortas desde un vértice fuente dado (origen) hasta todos los demás vértices en el grafo.
      \item[//] Utiliza dos arrays para almacenar la información:
      \begin{itemize}
         \item distancias: Almacena las distancias más cortas desde el vértice fuente a cada vértice en el grafo. Inicialmente, todas las distancias se establecen en $INT_MAX$ (infinito).
         \item predecesores: Almacena los predecesores en el camino más corto desde el vértice fuente a cada vértice. Inicialmente, todos los predecesores se establecen en -1.
      \end{itemize}
      \item[//] Utiliza una cola de prioridad para almacenar los vértices a explorar y, en cada iteración, selecciona el vértice con la distancia más corta.
      \item[//] Actualiza las distancias y los predecesores según las aristas y los pesos del grafo.
   \end{itemize}

   \item[5.] Función $dijkstra_thread$ (sin implementación):
   \begin{itemize}
      \item[//] Esta función es la que se ejecutará en paralelo en cada hilo para calcular Dijkstra en una porción del grafo. Sin embargo, la implementación de esta función está incompleta y deberás completarla para que realice el cálculo de Dijkstra en su porción del grafo.
   \end{itemize}

   \item[6.] Función Principal (main):
   \begin{itemize}
      \item[//] En main, se crea una instancia del grafo g con 20 vértices y se imprime su información.
      \item[//] Luego, se crea un número específico de hilos ($NUM_THREADS$) que ejecutarán la función $dijkstra_thread$ en paralelo en el mismo grafo.
      \item[//] Se espera a que todos los hilos terminen su ejecución antes de liberar la memoria asignada dinámicamente para el grafo y finalizar el programa.
   \end{itemize}
\end{itemize}

\section*{Pseudocódigo del Algotitmo}
\begin{verbatim}
    1. Inicializar todas las distancias en D con un valor infinito relativo, 
       ya que son desconocidas al principio, exceptuando la de a, 
       qué se debe colocar en 0, pues la distancia de a a si mismo sería 0. 
       C es copia de V        
    
    2. Para todo vértice i en C se establece [PI]= a.        
    
    3. Se obtiene el vértice s en C tal que no existe otro vértice w en C 
       tal que (D[w] < D[s]). 
       Para esto se envía un mensaje al nodo correspondiente y se regresa
       un mensaje de respuesta en donde se toma el tiempo y se le asigna a 
       su distancia correspondiente. De manera concurrente el nodo destino 
       realiza el mismo procedimiento para calcular su distancia a sus nodos
       vecinos que no han sido visitados.
    
        // En lugar de buscar el vértice con la distancia más corta 
        // iterativamente, ahora se utiliza una heap para mantener una lista 
        // de vértices no visitados, ordenada por la distancia más corta. Así 
        // encontrar el vértice n la distancia más corta en tiempo logarítmico.    
    
    4. Se elimina de C el vértice s. El vértice u se elimina del conjunto C.
    
    5. Para cada arista e en E de longitud l, 
       que une el vértice s con algún otro vértice t en C,
       Para cada arista que sale del vértice u, se verifica si la distancia 
       a través del vértice u es menor que la distancia actual del vértice t.
    
        - Si l + D[s] < D[t], entonces:
         // Si la distancia a través del vértice u es menor que la distancia 
         // actual del vértice t, entonces se actualiza la distancia del vértice t.
        - Se establece D[t] :=  l + D[s].
         // La distancia del vértice t se establece en la suma de la distancia 
         // del vértice u y el peso de la arista.
        - Se establece P[t] := s.
        // El predecesor del vértice t se establece en el vértice u.
    
    6. Se regresa al paso 4.
        // El algoritmo regresa al paso 4 y repite el proceso hasta que todos
        // los vértices hayan sido visitados.    
\end{verbatim}
    

\section*{Desarrollo}








\end{document}%---------------------- FIN DOCUMENTO---------------|
%-----------------------------------------------------------------|