\documentclass[a4paper,12pt]{article} 
\usepackage[utf8]{inputenc} % Acentos válidos sin problemas
\usepackage[spanish]{babel} % Idioma
\input{packet}


\begin{document}%----------------------INICIO DOCUMENTO------------|
%------------------------------------------------------------------|

\newpage

\input{portada}

\newpage

\begin{center}
    {\Large Algotitmo Dijkstra Distribuido}
\end{center}

\section*{Forma de compilar}

\section*{Funcionamiento}

\section*{Pseudocódigo del Algotitmo}
\begin{verbatim}
    1. Inicializar todas las distancias en D con un valor infinito relativo, 
       ya que son desconocidas al principio, exceptuando la de a, qué se debe 
       colocar en 0, pues la distancia de a a si mismo sería 0. C es copia de V
    
    2. Para todo vértice i en C se establece [PI]= a.
    
    3. Se obtiene el vértice s en C tal que no existe otro vértice w en C 
       tal que (D[w] < D[s]). Para esto se envía un mensaje al nodo correspondiente 
       y se regresa un mensaje de respuesta en donde se toma el tiempo y se le 
       asigna a su distancia correspondiente. De manera concurrente el nodo destino 
       realiza el mismo procedimiento para calcular su distancia a sus nodos vecinos
       que no han sido visitados.
    
    4. Se elimina de C el vértice s.
    
    5. Para cada arista e en E de longitud l, que une el vértice s con algún 
       otro vértice t en C,
    
    - Si l + D[s] < D[t], entonces:
      -- Se establece D[t] :=  l + D[s].
      -- Se establece P[t] := s.
      
    6. Se regresa al paso 4.
\end{verbatim}
    

\section*{Desarrollo}








\end{document}%---------------------- FIN DOCUMENTO---------------|
%-----------------------------------------------------------------|