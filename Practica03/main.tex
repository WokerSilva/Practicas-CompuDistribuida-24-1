\documentclass[a4paper,12pt]{article} 
\usepackage[utf8]{inputenc} % Acentos válidos sin problemas
\usepackage[spanish]{babel} % Idioma
\input{packet}


\begin{document}%----------------------INICIO DOCUMENTO------------|
%------------------------------------------------------------------|

\input{portada}
\newpage

\begin{center}
    {\huge Algoritmo abusón (bully)}
\end{center}

%%%%%%%%%%%%%%%%%%%%%%%%%%%%%%%%%%%%%%%
%%%%%%%%%%%%%%%%%%%%%%%%%%%%%%%%%%%%%%
%%%%%%%%%%%%%%%%%%%%%%%%%%%%%%%%%%%%
\section*{Ejecución del programa}
%%%%%%%%%%%%%%%%%%%%%%%%%%%%%%%%%%%%
%%%%%%%%%%%%%%%%%%%%%%%%%%%%%%%%%%%%%%
%%%%%%%%%%%%%%%%%%%%%%%%%%%%%%%%%%%%%%%

\subsection*{Compilar}
\begin{center}    
    Forma de Compilar 
\end{center}

\subsection*{Ejecutar}
\begin{center}
    Forma de Ejecutar
\end{center}


%%%%%%%%%%%%%%%%%%%%%%%%%%%%%%%%%%%%%%%
%%%%%%%%%%%%%%%%%%%%%%%%%%%%%%%%%%%%%%
%%%%%%%%%%%%%%%%%%%%%%%%%%%%%%%%%%%%
\section*{Elección distribuida}
%%%%%%%%%%%%%%%%%%%%%%%%%%%%%%%%%%%%
%%%%%%%%%%%%%%%%%%%%%%%%%%%%%%%%%%%%%%
%%%%%%%%%%%%%%%%%%%%%%%%%%%%%%%%%%%%%%%

\subsection*{Consideraciones}

\begin{itemize}
    \item Permite la caída de procesos durante la elección
     \begin{itemize}
        \item Utiliza \textit{timeouts} para detectar fallos de procesos
     \end{itemize}
    \item Supone comunicación fiable
    \item Cada proceso conoce qué procesos tienen identificadores mayores y puede comunicarse con ellos
\end{itemize}

\subsection*{Funcionamiento resumido}

\begin{enumerate}
    \item El convocante envía \textit{mensajes elección} a los procesos de $id$ mayor
    \item Si ninguno le responde, \textit{multidifunde} que es el nuevo coordinador
    \item Si alguno le responde, el convocante inicial queda en espera, y los procesos que responden
         inician un nuevo proceso de elección como convocantes (vuelta al paso 1)
\end{enumerate}

\subsection*{Tipos de mensaje}

\begin{itemize}
    \item \textbf{Elección:} anuncia un proceso de elección
    \item \textbf{Respuesta:} respuesta a un mensaje de elección
    \item \textbf{Coordinador:} anuncia la identidad del proceso elegido
\end{itemize}

\subsection*{Inicio del algoritmo}
\begin{itemize}
    \item La elección comienza cuando un proceso se da cuenta (debido a los \textit{timeouts}) de que el coordinador ha fallado
    \begin{itemize}
        \item Varios procesos pueden descubrirlo de forma concurrente
        \item Timeout = T = $2 \times T_{\text{Transmisión de un mensaje}} + T_{\text{procesado de un mensaje}}$
    \end{itemize}
    \item Si un proceso sabe que tiene el id (no fallido) más alto
    \begin{itemize}
        \item Se elige a sí mismo coordinador enviando el mensaje coordinador a todos los de identificador más bajo (proceso abusón)
    \end{itemize}
    \item Si no tiene el id (no fallido) más alto
    \begin{itemize}
        \item Manda un mensaje elección a todos los de id más alto y espera un mensaje respuesta
        \begin{itemize}
            \item Si tras un tiempoT no recibe ningún mensaje respuesta, ir al paso 1
            \item Si recibe un mensaje respuesta, espera un mensaje coordinador
            \begin{itemize}
                \item Si recibe un mensaje coordinador, fija su variable e; al id que está contenido en dicho mensaje
                \item Si no recibe ningún mensaje, comienza otra nueva elección                
            \end{itemize}
        \end{itemize}
    \end{itemize}
    \item Si un proceso se recupera o se lanza un proceso sustituto con el mismo id, éste comienza una nueva elección, aunque el coordinador actual esté funcionando
\end{itemize}


%%%%%%%%%%%%%%%%%%%%%%%%%%%%%%%%%%%%%%%
%%%%%%%%%%%%%%%%%%%%%%%%%%%%%%%%%%%%%%
%%%%%%%%%%%%%%%%%%%%%%%%%%%%%%%%%%%%
\section*{Funcionamiento}
%%%%%%%%%%%%%%%%%%%%%%%%%%%%%%%%%%%%
%%%%%%%%%%%%%%%%%%%%%%%%%%%%%%%%%%%%%%
%%%%%%%%%%%%%%%%%%%%%%%%%%%%%%%%%%%%%%%



%%%%%%%%%%%%%%%%%%%%%%%%%%%%%%%%%%%%%%%
%%%%%%%%%%%%%%%%%%%%%%%%%%%%%%%%%%%%%%
%%%%%%%%%%%%%%%%%%%%%%%%%%%%%%%%%%%%
\section*{Pseudocódigo del Algoritmo}
%%%%%%%%%%%%%%%%%%%%%%%%%%%%%%%%%%%%
%%%%%%%%%%%%%%%%%%%%%%%%%%%%%%%%%%%%%%
%%%%%%%%%%%%%%%%%%%%%%%%%%%%%%%%%%%%%%%


\begin{verbatim}
    Aquí va el código 
\end{verbatim}

%-----------------------------------------------------------------|
\end{document}%---------------------- FIN DOCUMENTO---------------|
%-----------------------------------------------------------------|