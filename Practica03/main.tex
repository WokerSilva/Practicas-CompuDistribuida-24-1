\documentclass[a4paper,12pt]{article} 
\usepackage[utf8]{inputenc} % Acentos válidos sin problemas
\usepackage[spanish]{babel} % Idioma
\input{packet}


\begin{document}%----------------------INICIO DOCUMENTO------------|
%------------------------------------------------------------------|

\input{portada}
\newpage

\begin{center}
    {\huge Algoritmo abusón (bully)}
\end{center}

%%%%%%%%%%%%%%%%%%%%%%%%%%%%%%%%%%%%%%%
%%%%%%%%%%%%%%%%%%%%%%%%%%%%%%%%%%%%%%
%%%%%%%%%%%%%%%%%%%%%%%%%%%%%%%%%%%%
\section*{Ejecución del programa}
%%%%%%%%%%%%%%%%%%%%%%%%%%%%%%%%%%%%
%%%%%%%%%%%%%%%%%%%%%%%%%%%%%%%%%%%%%%
%%%%%%%%%%%%%%%%%%%%%%%%%%%%%%%%%%%%%%%

\subsection*{Compilar}

\begin{small}
    \begin{verbatim}
mpicc Practica03_EdgarMontiel_CarlosCortes_MarcoSilva.c -o Practica03_EdgarMontiel_CarlosCortes_MarcoSilva
    \end{verbatim}            
\end{small}
\subsection*{Ejecutar}
\begin{verbatim}
    ./Practica03_EdgarMontiel_CarlosCortes_MarcoSilva
\end{verbatim}


%%%%%%%%%%%%%%%%%%%%%%%%%%%%%%%%%%%%%%%
%%%%%%%%%%%%%%%%%%%%%%%%%%%%%%%%%%%%%%
%%%%%%%%%%%%%%%%%%%%%%%%%%%%%%%%%%%%
\section*{Elección distribuida}
%%%%%%%%%%%%%%%%%%%%%%%%%%%%%%%%%%%%
%%%%%%%%%%%%%%%%%%%%%%%%%%%%%%%%%%%%%%
%%%%%%%%%%%%%%%%%%%%%%%%%%%%%%%%%%%%%%%

\subsection*{Consideraciones}

\begin{itemize}
    \item Permite la caída de procesos durante la elección
     \begin{itemize}
        \item Utiliza \textit{timeouts} para detectar fallos de procesos
     \end{itemize}
    \item Supone comunicación fiable
    \item Cada proceso conoce qué procesos tienen identificadores mayores y puede comunicarse con ellos
\end{itemize}

\subsection*{Funcionamiento resumido}

\begin{enumerate}
    \item El convocante envía \textit{mensajes elección} a los procesos de $id$ mayor
    \item Si ninguno le responde, \textit{multidifunde} que es el nuevo coordinador
    \item Si alguno le responde, el convocante inicial queda en espera, y los procesos que responden
         inician un nuevo proceso de elección como convocantes (vuelta al paso 1)
\end{enumerate}

\subsection*{Tipos de mensaje}

\begin{itemize}
    \item \textbf{Elección:} anuncia un proceso de elección
    \item \textbf{Respuesta:} respuesta a un mensaje de elección
    \item \textbf{Coordinador:} anuncia la identidad del proceso elegido
\end{itemize}


%%%%%%%%%%%%%%%%%%%%%%%%%%%%%%%%%%%%%%%
%%%%%%%%%%%%%%%%%%%%%%%%%%%%%%%%%%%%%%
%%%%%%%%%%%%%%%%%%%%%%%%%%%%%%%%%%%%
\section*{Funcionamiento}
%%%%%%%%%%%%%%%%%%%%%%%%%%%%%%%%%%%%
%%%%%%%%%%%%%%%%%%%%%%%%%%%%%%%%%%%%%%
%%%%%%%%%%%%%%%%%%%%%%%%%%%%%%%%%%%%%%%



%%%%%%%%%%%%%%%%%%%%%%%%%%%%%%%%%%%%%%%
%%%%%%%%%%%%%%%%%%%%%%%%%%%%%%%%%%%%%%
%%%%%%%%%%%%%%%%%%%%%%%%%%%%%%%%%%%%
\section*{Pseudocódigo del Algoritmo}
%%%%%%%%%%%%%%%%%%%%%%%%%%%%%%%%%%%%
%%%%%%%%%%%%%%%%%%%%%%%%%%%%%%%%%%%%%%
%%%%%%%%%%%%%%%%%%%%%%%%%%%%%%%%%%%%%%%

\begin{verbatim}
    Algoritmo Abusivo

    Variables Globales:
    - Mi_Id: identificador único del proceso
    - Coordinador_Actual: identificador del coordinador actual
    - Esperando_Coordinador: booleano que indica si el proceso 
       está esperando un mensaje de coordinador
    
    Inicio del Algoritmo:

    Declaración de una estructura de Nodo 
        - Define una estructura llamada Nodo que almacena información sobre un nodo, incluyendo su ID, su estado de estar vivo o muerto, el ID de su líder y un indicador de si está esperando una respuesta.
    
    Cuando un proceso detecta que el coordinador ha fallado debido a un timeout:
        - Iniciar la elección:
        
        Si Mi_Id es el identificador más alto no fallido:
            - Mi_Id es el nuevo coordinador
            - Enviar un mensaje de coordinador a todos los procesos con 
               identificadores más bajos (proceso abusón)
    
        Si Mi_Id no es el identificador más alto no fallido:
            - Enviar un mensaje elección a todos los procesos con 
               identificadores más altos
            - Esperar un mensaje de respuesta durante un tiempo T
            - Si no se recibe ningún mensaje de respuesta después de un tiempo T:
                - Regresar al paso de Inicio del Algoritmo
            - Si se recibe un mensaje de respuesta:
                - Esperar un mensaje de coordinador
                - Si se recibe un mensaje de coordinador:
                    - Coordinador_Actual = Id contenido en el mensaje de 
                       coordinador
                - Si no se recibe un mensaje de coordinador:
                    - Iniciar una nueva elección

    Función simulaTimeout
        - Esta función simula un timeout con una probabilidad del 30% de fallar. Devuelve true si se produce un timeout y false en caso contrario.
    
    Si un proceso se recupera o se lanza un proceso sustituto con el mismo Mi_Id:
        - Iniciar una nueva elección, independientemente de si el coordinador 
           actual está funcionando

    Función main:
    - En la función principal, se inicializa la semilla aleatoria y se inicia MPI.
    - Se crea un arreglo de estructuras Nodo llamado nodos, y se inicializan los nodos con sus IDs, estado de estar vivo y líderes.
    - Se define un valor para myId, que representa el ID del nodo actual.
    - Se imprime un mensaje para indicar el inicio de la elección.
    - Se llama a la función iniciarElección para el nodo actual.
    - Luego, se imprime el resultado de la elección para cada nodo en el sistema.
    - Finalmente, se llama a MPI_Finalize para finalizar la comunicación MPI.
\end{verbatim}

%-----------------------------------------------------------------|
\end{document}%---------------------- FIN DOCUMENTO---------------|
%-----------------------------------------------------------------|