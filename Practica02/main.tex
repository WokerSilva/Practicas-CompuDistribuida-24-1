\documentclass[a4paper,12pt]{article} 
\usepackage[utf8]{inputenc} % Acentos válidos sin problemas
\usepackage[spanish]{babel} % Idioma
\input{packet}


\begin{document}%----------------------INICIO DOCUMENTO------------|
%------------------------------------------------------------------|

\newpage
\input{portada}
\newpage

\begin{center}
    {\huge Consenso}
\end{center}

%%%%%%%%%%%%%%%%%%%%%%%%%%%%%%%%%%%%%%%
%%%%%%%%%%%%%%%%%%%%%%%%%%%%%%%%%%%%%%
%%%%%%%%%%%%%%%%%%%%%%%%%%%%%%%%%%%%
\section*{Ejecución del programa}
%%%%%%%%%%%%%%%%%%%%%%%%%%%%%%%%%%%%
%%%%%%%%%%%%%%%%%%%%%%%%%%%%%%%%%%%%%%
%%%%%%%%%%%%%%%%%%%%%%%%%%%%%%%%%%%%%%%

\subsection*{Compilar}
\begin{center}    
    mpicc Practica02$\_$EdgarMontiel$\_$CarlosCortes$\_$MarcoSilva.c -o generales
\end{center}

\subsection*{Ejecutar}
\begin{center}
    $.\slash$generales
\end{center}

%%%%%%%%%%%%%%%%%%%%%%%%%%%%%%%%%%%%%%%
%%%%%%%%%%%%%%%%%%%%%%%%%%%%%%%%%%%%%%
%%%%%%%%%%%%%%%%%%%%%%%%%%%%%%%%%%%%
\section*{Funcionamiento}
%%%%%%%%%%%%%%%%%%%%%%%%%%%%%%%%%%%%
%%%%%%%%%%%%%%%%%%%%%%%%%%%%%%%%%%%%%%
%%%%%%%%%%%%%%%%%%%%%%%%%%%%%%%%%%%%%%%

\begin{enumerate}
  \item \textbf{Definición de constantes y estructuras:}
  \begin{itemize}
      \item Se definen constantes, como el número total de generales (\texttt{NUM\_GENERALES}), el número de generales traidores (\texttt{NUM\_TRAIDORES}), el número máximo de rondas (\texttt{MAX\_RONDAS}), y el número de generales traidores tolerados (\texttt{F}).
      \item Se define una estructura \texttt{General} que almacena información sobre cada general, incluyendo su identificación (\texttt{id}), si es traidor (\texttt{es\_traidor}), su voto (\texttt{voto}), su mensaje (\texttt{mensaje}), y su estrategia de voto (\texttt{estrategia}).
  \end{itemize}
  
  \item \textbf{Función para determinar si la votación es válida (\texttt{esVotacionValida}):}
  \begin{itemize}
      \item Esta función cuenta los votos a favor de atacar y retirarse y verifica si se alcanza la mayoría requerida para validar la votación.
  \end{itemize}
  
  \item \textbf{Función para realizar una ronda de comunicación (\texttt{realizarRonda}):}
  \begin{itemize}
      \item En cada ronda, los generales eligen aleatoriamente si votar por atacar (1) o retirarse (0) según su estrategia.
      \item El voto se almacena en el campo \texttt{voto} y se copia en el campo \texttt{mensaje} del general.
  \end{itemize}
  
  \item \textbf{Función para elegir un rey (\texttt{elegirRey}):}
  \begin{itemize}
      \item Esta función determina cuál de los generales no traidores tiene el ID más alto y lo elige como rey.
  \end{itemize}
  
  \item \textbf{Función para imprimir el resultado de una ronda (\texttt{imprimirResultado}):}
  \begin{itemize}
      \item Esta función imprime el número de la ronda actual, muestra información sobre cada general (ID, si es traidor y su voto), y verifica si la votación es válida llamando a la función \texttt{esVotacionValida}.
  \end{itemize}
  
  \item \textbf{Función principal (\texttt{main}):}
  \begin{itemize}
      \item Se inicializa el generador de números aleatorios.
      \item Se crea un arreglo de generales y se establecen sus atributos iniciales, como ID, si son traidores, voto indefinido, mensaje indefinido y estrategia de voto aleatoria.
  \end{itemize}
  
  \item \textbf{Ciclo principal (\texttt{while}):}
  \begin{itemize}
      \item Se realiza un ciclo de rondas de comunicación mientras no se alcance un consenso o se supere el número máximo de rondas definido en \texttt{MAX\_RONDAS}.
      \item En cada ronda, se realiza una ronda de comunicación aleatoria (\texttt{realizarRonda}) y se muestra el resultado (\texttt{imprimirResultado}).
      \item Se verifica si la votación es válida. Si es válida, se muestra un mensaje y se rompe el ciclo.
      \item Si no se alcanza un consenso, se elige un nuevo rey entre los generales no traidores (\texttt{elegirRey}).
  \end{itemize}
  
  \item Si se supera el límite de rondas definido en \texttt{MAX\_RONDAS}, se muestra un mensaje indicando que se alcanzó el límite de rondas sin consenso.
\end{enumerate}

%%%%%%%%%%%%%%%%%%%%%%%%%%%%%%%%%%%%%%%
%%%%%%%%%%%%%%%%%%%%%%%%%%%%%%%%%%%%%%
%%%%%%%%%%%%%%%%%%%%%%%%%%%%%%%%%%%%
\section*{Pseudocódigo del Algoritmo}
%%%%%%%%%%%%%%%%%%%%%%%%%%%%%%%%%%%%
%%%%%%%%%%%%%%%%%%%%%%%%%%%%%%%%%%%%%%
%%%%%%%%%%%%%%%%%%%%%%%%%%%%%%%%%%%%%%%

\subsection*{Algoritmo del Rey}
\begin{verbatim}
1. Definir las constantes:
   - NÚMERO_DE_GENERALES: número total de generales
   - NÚMERO_DE_TRAIDORES: número de generales traidores
   - F: número de generales traidores tolerados

2. Crear una estructura General con los siguientes campos:
   - id (entero): identificador del general
   - es_traidor (booleano): verdadero si el general es traidor, falso si es leal
   - voto (entero): 0 para retirada, 1 para ataque
   - mensaje (entero): mensaje enviado por el general en la ronda actual

3. Inicializar una lista de generales con NÚMERO_DE_GENERALES elementos.

4. Inicializar una variable REY con un valor aleatorio en el rango
               [0, NÚMERO_DE_GENERALES - 1]
   - Esto selecciona aleatoriamente a un general como el Rey sin 
     que los demás lo sepan.

5. Para cada general en la lista de generales:
   - Asignar un id único al general.
   - Determinar si el general es traidor (F generales serán traidores,
     incluyendo el Rey).
   - Inicializar el voto y el mensaje del general.

6. En cada ronda:
   - Cada general, incluido el Rey, elige su voto (0 para retirada, 1 para ataque) 
     de acuerdo a su estrategia.

7. Calcular el resultado de la ronda:
   - Inicializar las variables votos_ataque y votos_retirada a 0.
   - Para cada general en la lista de generales:
     - Si el general no es traidor:
       - Incrementar votos_ataque o votos_retirada según el voto del general.
     - Si el general es traidor:
       - Tomar el voto del general según su estrategia.

8. Verificar si la votación es válida:
   - Calcular la mayoría requerida como "(NÚMERO_DE_GENERALES / 2) + F".
   - Si votos_ataque >= mayoría o votos_retirada >= mayoría, la votación es válida.

9. Imprimir el resultado de la ronda y si la votación es válida o no.

10. Repetir las rondas hasta que se alcance un resultado válido o se llegue a un 
    límite de rondas.

11. Si se supera el límite de rondas, se considera que no hay consenso y se imprime
    un mensaje indicando la falta de consenso.

12. Finalizar el algoritmo.

\end{verbatim}



%%%%%%%%%%%%%%%%%%%%%%%%%%%%%%%%%%%%%%%
%%%%%%%%%%%%%%%%%%%%%%%%%%%%%%%%%%%%%%
%%%%%%%%%%%%%%%%%%%%%%%%%%%%%%%%%%%%%
\section*{Desarrollo}
%%%%%%%%%%%%%%%%%%%%%%%%%%%%%%%%%%%%
%%%%%%%%%%%%%%%%%%%%%%%%%%%%%%%%%%%%%%
%%%%%%%%%%%%%%%%%%%%%%%%%%%%%%%%%%%%%%%







\end{document}%---------------------- FIN DOCUMENTO---------------|
%-----------------------------------------------------------------|