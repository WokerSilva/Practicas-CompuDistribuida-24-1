\documentclass[a4paper,12pt]{article} 
\usepackage[utf8]{inputenc} % Acentos válidos sin problemas
\usepackage[spanish]{babel} % Idioma
\input{packet}


\begin{document}%----------------------INICIO DOCUMENTO------------|
%------------------------------------------------------------------|

\newpage
\input{portada}
\newpage

\begin{center}
    {\huge Consenso}
\end{center}

%%%%%%%%%%%%%%%%%%%%%%%%%%%%%%%%%%%%%%%
%%%%%%%%%%%%%%%%%%%%%%%%%%%%%%%%%%%%%%
%%%%%%%%%%%%%%%%%%%%%%%%%%%%%%%%%%%%
\section*{Ejecución del programa}
%%%%%%%%%%%%%%%%%%%%%%%%%%%%%%%%%%%%
%%%%%%%%%%%%%%%%%%%%%%%%%%%%%%%%%%%%%%
%%%%%%%%%%%%%%%%%%%%%%%%%%%%%%%%%%%%%%%

\subsection*{Compilar}
\begin{center}
    gcc -o programa Practica02$\_$EdgarMontiel$\_$CarlosCortes$\_$MarcoSilva.c
\end{center}

\subsection*{Ejecutar}
\begin{center}
    $.\slash$programa
\end{center}

%%%%%%%%%%%%%%%%%%%%%%%%%%%%%%%%%%%%%%%
%%%%%%%%%%%%%%%%%%%%%%%%%%%%%%%%%%%%%%
%%%%%%%%%%%%%%%%%%%%%%%%%%%%%%%%%%%%
\section*{Funcionamiento}
%%%%%%%%%%%%%%%%%%%%%%%%%%%%%%%%%%%%
%%%%%%%%%%%%%%%%%%%%%%%%%%%%%%%%%%%%%%
%%%%%%%%%%%%%%%%%%%%%%%%%%%%%%%%%%%%%%%




%%%%%%%%%%%%%%%%%%%%%%%%%%%%%%%%%%%%%%%
%%%%%%%%%%%%%%%%%%%%%%%%%%%%%%%%%%%%%%
%%%%%%%%%%%%%%%%%%%%%%%%%%%%%%%%%%%%
\section*{Pseudocódigo del Algoritmo}
%%%%%%%%%%%%%%%%%%%%%%%%%%%%%%%%%%%%
%%%%%%%%%%%%%%%%%%%%%%%%%%%%%%%%%%%%%%
%%%%%%%%%%%%%%%%%%%%%%%%%%%%%%%%%%%%%%%

\subsection*{Algoritmo del Rey}
\begin{verbatim}
1. Definir las constantes:
   - NÚMERO_DE_GENERALES: número total de generales
   - NÚMERO_DE_TRAIDORES: número de generales traidores
   - F: número de generales traidores tolerados

2. Crear una estructura General con los siguientes campos:
   - id (entero): identificador del general
   - es_traidor (booleano): verdadero si el general es traidor, falso si es leal
   - voto (entero): 0 para retirada, 1 para ataque
   - mensaje (entero): mensaje enviado por el general en la ronda actual

3. Inicializar una lista de generales con NÚMERO_DE_GENERALES elementos.

4. Inicializar una variable REY con un valor aleatorio en el rango
               [0, NÚMERO_DE_GENERALES - 1]
   - Esto selecciona aleatoriamente a un general como el Rey sin 
     que los demás lo sepan.

5. Para cada general en la lista de generales:
   - Asignar un id único al general.
   - Determinar si el general es traidor (F generales serán traidores,
     incluyendo el Rey).
   - Inicializar el voto y el mensaje del general.

6. En cada ronda:
   - Cada general, incluido el Rey, elige su voto (0 para retirada, 1 para ataque) 
     de acuerdo a su estrategia.

7. Calcular el resultado de la ronda:
   - Inicializar las variables votos_ataque y votos_retirada a 0.
   - Para cada general en la lista de generales:
     - Si el general no es traidor:
       - Incrementar votos_ataque o votos_retirada según el voto del general.
     - Si el general es traidor:
       - Tomar el voto del general según su estrategia.

8. Verificar si la votación es válida:
   - Calcular la mayoría requerida como "(NÚMERO_DE_GENERALES / 2) + F".
   - Si votos_ataque >= mayoría o votos_retirada >= mayoría, la votación es válida.

9. Imprimir el resultado de la ronda y si la votación es válida o no.

10. Repetir las rondas hasta que se alcance un resultado válido o se llegue a un 
    límite de rondas.

11. Si se supera el límite de rondas, se considera que no hay consenso y se imprime
    un mensaje indicando la falta de consenso.

12. Finalizar el algoritmo.

\end{verbatim}



%%%%%%%%%%%%%%%%%%%%%%%%%%%%%%%%%%%%%%%
%%%%%%%%%%%%%%%%%%%%%%%%%%%%%%%%%%%%%%
%%%%%%%%%%%%%%%%%%%%%%%%%%%%%%%%%%%%%
\section*{Desarrollo}
%%%%%%%%%%%%%%%%%%%%%%%%%%%%%%%%%%%%
%%%%%%%%%%%%%%%%%%%%%%%%%%%%%%%%%%%%%%
%%%%%%%%%%%%%%%%%%%%%%%%%%%%%%%%%%%%%%%







\end{document}%---------------------- FIN DOCUMENTO---------------|
%-----------------------------------------------------------------|